%=========================================================

%\chapter{Glosario de términos}
%\label{cap:glosario}

%%%%%%%%%%%%% DEFINICIONES %%%%%%%%%%%%%
%\section{Definiciones}
%    \begin{itemize} 
        %%%%% A %%%%%

 %       \item \textbf{Análisis:} Es la parte del proceso de desarrollo de software cuyo propósito principal es realizar un modelo del dominio del problema. El análisis hace foco en qué hacer, el diseño hace foco en cómo hacerlo.
    
        %%%%% D %%%%%
        %%%%% E %%%%%
 %       \item \textbf{Especificación:} Es un informe de acuerdo entre el implementador y el usuario.
  %      \item \textbf{Especificación de requerimientos:} Es aquella que establece un acuerdo entre el usuario y el desarrollador del sistema.
        %%%%% G %%%%%

        %%%%% I %%%%%

        %%%%% M %%%%%
   %     \item \textbf{Modelo:} Es una abstracción semánticamente consistente de un sistema.
        %%%%% P %%%%%
        %%%%% R %%%%%
    %    \item \textbf{Requerimiento:} Es una característica, propiedad o comportamiento deseado para un sistema.
        %%%%% S %%%%%
       
    %\end{itemize}
    
%%%%%%%%%%%%% ACRÓNIMOS %%%%%%%%%%%%%
%\section{Acrónimos}
 %   \begin{itemize} 
  %      \item \textbf{ESCOM:} Escuela Superior de Cómputo.
   % 	\item  \textbf{IPN:} Instituto Politécnico Nacional.
    %	\item \textbf{SAT:} Servicio de Administración Tributaria.
    %	\item \textbf{IMSS:} Instituto Mexicano del Seguro Social.
    %	\item \textbf{SEP:} Secretaria de Educación Pública.
    %	\item \textbf{CURP:} Clave Única de Registro de Población.
    %\end{itemize}


