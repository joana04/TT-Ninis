\begin{UseCase}{CU3.09}{Editar documentación de la Empresa}{
    Permite al usuario \cdtRef{Actor: Empresa}{Empresa}  editar su documentación registrada en el sistema, el usuario presiona el botón 
    \IUbutton{Editar} en la pantalla \IUref{IU3.06}{Visualización de Documentos de Empresa}, el sistema recupera la información del usuario y la muestra en la pantalla \IUref{IU3.09}{Edición de Documentos de la Empresa}, esta pantalla le permite al usuario editar sus datos en el sistema, una vez que selecciono los archivos que desea  cambiar puede presionar el botón \IUbutton{Cancelar} para cancelar la edición de su información o el botón \IUbutton{Guardar} para almacenar los cambios en el sistema. 
    
    
    \bigskip
}
		\UCitem{Versión}{1.0}
		\UCitem {Actor}{\cdtRef{Actor: Empresa}{Empresa} }
		\UCitem{Propósito}{El representante de la \cdtRef{Actor: Empresa}{Empresa}  podrá editar sus datos en el sistema.}
		\UCitem{Entradas}{Información ingresada por el usuario:
			\begin{itemize}	
  		        \item Campo o campos que desea editar.
			\end{itemize}
	     }
		\UCitem{Salidas}{
			\begin{itemize}
    			
    			\item \MSGref{MSG1.01}{Registro exitoso.}
    			\item \MSGref{MSG1.03}{Operación cancelada.}
    			\item \MSGref{MSG2.04}{Campos obligatorios vacíos.}
    			\item \MSGref{MSG2.05}{Formato de archivo incorrecto.}
    			\item \MSGref{MSG2.06}{Excede tamaño de archivo permitido.}
    		\end{itemize}}
		\UCitem{Precondiciones}{
    		Usuario registrado en el sistema.}
		\UCitem{Postcondiciones}{
		    \begin{itemize}
    		    \item Se generan cambios en el registro del usuario en el sistema.
    		 \end{itemize}}
\end{UseCase}
	
	%-------------------------------------- COMIENZA descripción Trayectoria Principal
	\begin{UCtrayectoria}{Editar información }

        \UCpaso[\UCactor]Presiona el botón \IUbutton{Editar} de la pantalla \IUref{IU3.06}{Visualización de Documentos de Empresa}.\Trayref{A-CU3.09}
        \UCpaso[\UCsist] Obtiene la información del usuario.
		\UCpaso[\UCsist] Muestra la pantalla \IUref{IU3.09}{Edición de Documentos de la Empresa}. 
		\UCpaso[\UCactor] Ingresa los datos que desee editar de la pantalla \IUref{IU3.09}{Edición de Documentos de la Empresa}.
		\UCpaso[\UCactor] Presiona el botón \IUbutton{Guardar}. \Trayref{E-CU3.09}
		\UCpaso[\UCsist] Verifica que los campos estén con base en la regla de negocio \hyperlink{RN2}{RN-2 Campo Obligatorio}. \Trayref{B-CU3.09}
	    \UCpaso[\UCsist] Verifica que todos los archivos adjuntos estén con base en la regla de negocio \hyperlink{RN5}{RN-5 Adjuntar un archivo válido}. \Trayref{C-CU3.09}
	    \UCpaso[\UCsist] Verifica que todos los archivos adjuntos estén con base en la regla de negocio \hyperlink{RN6}{RN-6 Tamaño de archivo incorrecto}. \Trayref{D-CU3.09}
		
        \UCpaso[\UCsist] Actualiza la información del solicitante en el sistema.
	    \UCpaso[\UCsist] Muestra la pantalla \IUref{IU3.00}{Inicio de Empresa}.
	    \UCpaso[\UCsist] Valida con el servicio web del SAT(Servicio de Administración Tributaria ) la STPS(Secretaria de Trabajo y Previsión Social ).
	
	\end{UCtrayectoria}
	

	%---- A Cerrar
	%---- B Campos vacíos
	%---- C CURP invalido
	%---- D No. de seguro social invalido
	%---- E Cancelar
	

	
	

	%-------------------------------------Trayectorias alternativas A --> Cancelar
	\begin{UCtrayectoriaA}{A-CU3.09}{Presiona el botón \IUbutton{Cerrar} de la pantalla \IUref{IU3.06}{Visualización de Documentos de Empresa}}{A}
		\UCpaso[\UCsist] Muestra la pantalla \IUref{IU3.00}{Inicio de Empresas}.
		\item[- -] - - {\em Fin del caso de uso.} 
	\end{UCtrayectoriaA}

	%-------------------------------------Trayectorias alternativas B --> Campos vacíos
	\begin{UCtrayectoriaA}{B-CU3.09}{Campos vacíos.}{B}
	    \UCpaso[\UCsist]Muestra el mensaje \MSGref{MSG2.04}{Campos obligatorios vacíos}en la pantalla \IUref{IU3.09}{Edición de Documentos de la Empresa}.
	    \item[- -] - - {\em Regresa al punto número 4 de la trayectoria principal.}
	\end{UCtrayectoriaA}

%------------------------------------- trayectoria alternativa C --> Formato incorrecto 
	
	\begin{UCtrayectoriaA}{C-CU3.09}{Formato incorrecto.}{C}
	    \UCpaso[\UCsist]Muestra el mensaje \MSGref{MSG2.05}{Formato de archivo incorrecto} en la pantalla \IUref{IU3.09}{Edición de Documentos de la Empresa}.
	    \item[- -] - - {\em Regresa al punto número 4 de la trayectoria principal.}
	\end{UCtrayectoriaA}

%------------------------------------------Trayectoria alternativa D-->Tamaño incorrecto
	\begin{UCtrayectoriaA}{C-CU3.09}{Tamaño de archivo incorrecto.}{C}
	    \UCpaso[\UCsist]Muestra el mensaje \MSGref{MSG2.06}{Excede tamaño de archivo permitido} en la pantalla \IUref{IU3.09}{Edición de Documentos de la Empresa}.
	    \item[- -] - - {\em Regresa al punto número 4 de la trayectoria principal.}
	\end{UCtrayectoriaA}
	
    
	 %-------------------------------------Trayectorias alternativas E--> Cancelar
	\begin{UCtrayectoriaA}{E-CU3.08}{Presiona el botón \IUbutton{Cancelar} de la pantalla \IUref{IU3.09}{Edición de Documentos de la Empresa}}{E}
		\UCpaso[\UCsist] Muestra el mensaje \MSGref{MSG0.03}{Operación cancelada} en la pantalla \IUref{IU3.00}{Inicio de Empresas}.
		\item[- -] - - {\em Fin del caso de uso.} 
	\end{UCtrayectoriaA}


 