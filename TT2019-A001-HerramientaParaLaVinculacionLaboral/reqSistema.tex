\chapter{Requerimientos del sistema}
\label{cap:reqSist}

\section{Notación, símbolos y convenciones utilizadas}

	Los requerimientos funcionales utilizan una clave RFX, donde:
	
\begin{description}
	\item[RFS] Es la clave para todos los {\bf R}equerimientos {\bf F}uncionales del {\bf S}istema.
	\item[RFU] Es la clave para todos los {\bf R}equerimientos {\bf F}uncionales del {\bf U}suario.
	\item[RFA] Es la clave para todos los {\bf R}equerimientos {\bf F}uncionales del {\bf A}dministrador.
	\item[X] Es un número consecutivo: 1, 2, 3, ...
\end{description}

	Además, para los requerimientos funcionales se usan las abreviaciones que se muestran en la tabla~\ref{tbl:leyendaRF}.
\begin{table}[hbtp!]
	\begin{center}
    \begin{tabular}{|r l|}
	    \hline
    	{\footnotesize Id} & {\footnotesize\em Identificador del requerimiento.}\\
    	{\footnotesize Pri.} & {\footnotesize\em Prioridad}\\
    	{\footnotesize Ref.} & {\footnotesize\em Referencia a los Requerimientos de usuario.}\\
    	{\footnotesize MA} & {\footnotesize\em Prioridad Muy Alta.}\\
    	{\footnotesize A} & {\footnotesize\em Prioridad Alta.}\\
    	{\footnotesize M} & {\footnotesize\em Prioridad Media.}\\
    	{\footnotesize B} & {\footnotesize\em Prioridad Baja.}\\
    	{\footnotesize MB} & {\footnotesize\em Prioridad Muy Baja.}\\
		\hline
    \end{tabular} 
    \caption{Leyenda para los requerimientos funcionales.}
    \label{tbl:leyendaRF}
	\end{center}
\end{table}

\newpage

\section{Requerimientos funcionales}
Los requerimiento funcionales describen lo que el sistema debe hacer.\\

Son declaraciones de los servicios que debe proporcionar el sistema, de la manera en que éste debe reaccionar con determinadas entradas y situaciones, también pueden declarar lo que el sistema no debe hacer \cite{Req}

\begin{table}[htbp!]
	\begin{requerimientos}
	    \FRitem{RFS1}{Inicio de sesión}{El sistema permitirá el manejo de distintos tipos de sesiones de acuerdo al tipo de usuario.}{A}{RU}
	    \FRitem{RFS2}{Registro de becarios}{El sistema permitirá el registro, consulta, actualización y eliminación de los usuarios \cdtRef{Actor: Becario}{Becario} con el fin de tener registro de los solicitantes.}{A}{RU}
	   	\FRitem{RFS3}{Registro de empresas}{El sistema permitirá el registro, consulta, actualización y eliminación las \cdtRef{Actor: Empresa}{Empresa} que se encuentren ofertando vacantes.}{A}{RU}
		\FRitem{RFU1}{Registro de vacantes}{El usuario desea registrar, consultar, actualizar y eliminar vacantes ofertadas.}{A}{RU}

		\FRitem{RFA1}{Gestionar usuarios}{El administrador desea gestionar y visualizar la informacion de los usuario registrados en el sistema.}{A}{RU}
		\FRitem{RFA2}{Gestionar archivos}{El administrador desea gestionar los archivos registrados en el sistema.}{A}{RU}
	\end{requerimientos}
    \caption{Requerimientos funcionales.}
    \label{tb1:reqFunc}
\end{table}


\newpage

\section{Requerimientos no funcionales} 

Son restricciones de los servicios o funciones ofrecidos por el sistema. Incluyen restricciones de tiempo, proceso de desarrolloo y estándares\cite{Req}.

\begin{table}[htbp!]
	\begin{requerimientosnf}
	    %% -Requerimientos funcionales de sistema
	    \FRitem{RNF1}{Portabilidad}{El sistema estará disponible mediante una aplicacion web. Estará soportada con el siguiente navegador web:
	            \begin{itemize}
	                \item Google Chrome.
	            \end{itemize}
	            Según las estadisticas de \cite{statcounter} el 63.72 \% de los usuarios utilizan Chrome para navegar, a demás que es soportado por Linux, Windows, etc.
	            \newline
	            Como versión utilizaremos la mas reciente a la que estamos realizando el TT (Versión 77.0.3865.90).
	        }{A}{RU}
	    %ado hasta por 400 usuario, (se comprobará a través de pruebas de estrés) siempre y cuando se tenga una conexion estable a Internet.}{A}{RU}
	    \FRitem{RNF2}{Fiabilidad}{
	        \begin{itemize}
	            \item El sistema deberá trabajar de manera correcta bajo condiciones de uso especificadas en los Caosos de Uso.
	            \item El sistema trabajará correctamente con una conexión estable a internet.
	            \item No debe haber perdida de información cuando se realiza una transacción y haya una falla en el servidor.
	        \end{itemize}
	        %aqui estaba lo de no exigir de más al sistema
	       
	        Nota: Se refiere condición estable cuando tenemos conexión a internet con un ancho de banda mayor a 1000 kb/s por medio de cable Ethernet, sin ningun tipo de fallo de hardware.}{A}{RU}
		\FRitem{RNF3}{Mantenibilidad}{Todo el sistema estará documentado y estructurado.
		\newline
		Procesos, cosos de uso y código estructurados de una manera consistente e integral (con la documentación correspondiente) previendo la facilidad del mantenimiento a futuro.}{A}{RU}
		
	\end{requerimientosnf}
    \caption{Requerimientos no funcionales del sistema.}
    \label{tb2:reqNoFunc}
\end{table}
 %Nota: Se considera condición normal las consultas o las transacciones que pueda realizar.
%	        \newline
\newpage
%%--------------
\subsection{Restricciones de construcción}
\begin{table}[htbp!]
	\begin{restcon}
	    %% -Requerimientos funcionales de sistema
	    \RCitem{RC1}{Manejador de Base de Datos}{El sistema usa el  manejado de base de datos MySQL.}
	    \RCitem{RC2}{Sistema Operativo del servidor}{El servidor esta montado en Google Cloud con un SO Windows Server 2012.}
	    \RCitem{RC3}{Memoria RAM del servidor}{El servidor tiene una memoria RAM de 16384 MB.}
	    \RCitem{RC4}{CPU del servidor}{El servidor tiene un CPU Dual 2.8 GHz Intel Xenon(R) (Hyper-Threaded).}
		\RCitem{RC5}{Sistema Operativo en el sistema}{Al utilizar Google Chrome podemos asegurar que podra utilizar el sistema en cuelaquier Sistema Operativo. }
		
\end{restcon}
    \caption{Restricciones de construcción del sistema.}
    \label{tb3:RestriccionesDeConstruccion}
\end{table}
