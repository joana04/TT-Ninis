\section{Contexto de trabajo}

%%Se describe el área donde se esta trabajando y en qué 

%%Se propone tener una cobertura en la (zona metropolitana del país o CDMX), teniendo como población objetivo los jóvenes que no se encuentren trabajando o preparándose profesionalmente para incorporarse a la Población Económicamente Activa (PEA) al momento de registrarse en el sistema. 
Un joven debería ser parte de la fuerza laboral o bien, se debería preparar para entrar a ella. De no ser así —es decir, si ese joven no tiene un empleo, ya sea formal o informal, y tampoco se está preparando en alguna institución educativa para ingresar a la fuerza laboral—, se le clasifica como NINI. \\

Hoy día la tasa de ninis en México esta incrementado llegando a una cifra de 3.9 millones de personas según el \cite{INEGI} \cite{OECD2}, generando un costo anual de 194,000 millones de pesos.

 En Parametría se reportan los resultados de una encuesta de opinión en viviendas y donde 58 por ciento de los entrevistados opina que para los ninis resulta más atractivo entrar a las filas del narcotráfico que conseguir un trabajo o asistir a la escuela. \cite{Parametria} Además, en su análisis, Escobedo \cite{JEB} menciona que 80 por ciento de los ninis ha participado en actos de violencia, aunado a esto una encuesta realizada por la firma OCCMundial ocho de cada diez jóvenes mexicanos no se inscribieron a una universidad y 42 por ciento no lo hizo porque no pudo pagar una licenciatura de estudios presenciales por otra parte seis de cada diez jóvenes abandonaron sus estudios superiores por falta de dinero. \cite{OCC} \cite{Forbes_Universidad} \\
 \bigskip
