%========================================================
% Regla del Negocio #7 - Formato de CP
%-----------------------------------------------
\section{Regla de Negocio 7.- Formato de Código Postal}

\begin{BussinesRule}{RN7}{Formato de Código Postal}
	\BRitem[Tipo:] Regla estructural. 
				% Otras opciones para tipo: 
				% - Regla de integridad referencial o estructural. 
				% - Regla de operación, (calcular o determinar un valor.).
				% - Regla de inferencia de un hecho.
	\BRitem[Clase:] Habilitadora. 
				% Otras opciones para clase: Habilitadora, Cronometrada, Ejecutiva.
	\BRitem[Nivel:] Control. % Otras opciones para nivel: Control, Influencia.
	\BRitem[Descripción:]Clave numérica compuesta por cinco dígitos que identifica y ubica un área geográfica del país y la oficina postal que la sirve, para facilitar al correo, el encaminamiento, la distribución y el reparto de la materia postal \cite{CP}. Su denominación abreviada será C.P.

	\BRitem[Motivación:] Registrar en el sistema CPs con estructuras correctas.
	\BRitem[Sentencia:] $\forall\ x \in CP \Rightarrow   [0-9]{1}[1-9]{1}[0-9]{3}$
	\BRitem[Ejemplo positivo:] Cumplen la regla de negocio los siguientes CPs:
        \begin{itemize}
			\item 57630
			\item 45678
        \end{itemize}
	\BRitem[Ejemplo negativo:] No cumplen la regla de negocio los siguientes CP:
		\begin{itemize}
        	\item 00564
			\item O54t6
        	
    \end{itemize}
\end{BussinesRule}



